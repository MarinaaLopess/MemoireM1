Actuellement, il a été démontré que le bruit OCR, les informations contextuelles transphrastiques et le \textit{Truecasing} ont un impact sur les performances des systèmes de Reconnaissance d'Entités Nommées (NER). Parmi ces facteurs, le bruit OCR constitue un problème clé dans le traitement des textes numérisés. En effet, les études récentes indiquent que le taux d’erreur des systèmes NER n'augmente pas de manière significative dans les textes bruités, mais que la reconnaissance des entités rares (telles que les mots n'apparaissant qu'une seule fois, ou \textit{hapax}) reste un défi majeur. Toutefois, bien que la robustesse des systèmes NER face au bruit OCR ait été validée dans l'analyse de livres français, la variabilité linguistique et la complexité de la mise en page nécessitent encore une attention soutenue. Par ailleurs, en ce qui concerne l'exploitation du contexte transphrastique, les recherches montrent que l'introduction de phrases multiples en entrée des modèles BERT, combinée à une stratégie de vote majoritaire contextuel (Contextual Majority Voting, CMV), permet d'atteindre des performances optimales pour les tâches NER multilingues. Enfin, le \textit{Truecasing} (rétablissement de la casse correcte) s'est également avéré capable d'améliorer le score F1 du NER jusqu'à 26\,\%, avec une efficacité particulièrement marquée dans les textes issus de la reconnaissance vocale (ASR).

\parallèlement
Parallèlement, l'adaptabilité des systèmes NER à différents domaines constitue un axe de recherche majeur. Dans le domaine littéraire, des chercheurs travaillant sur des œuvres néerlandaises ont ainsi pu analyser fonctionnellement les noms propres en élargissant les catégories d'entités — par exemple, en distinguant les entités fictives des entités réelles — et en adaptant les outils NER existants. De plus, la création du corpus \textit{PPORTAL\_ner} pour la littérature portugaise a permis de combler une lacune importante en matière de ressources linguistiques. Les études montrent qu'un \textit{fine-tuning} spécifique au domaine améliore significativement les performances des modèles NER appliqués à ce type de textes, confirmant ainsi l'importance de l'adaptation contextuelle. Dans une perspective similaire, les travaux d'Eleni Kogkitsidou sur des textes en ancien français ont révélé que la normalisation manuelle optimise nettement la reconnaissance d'entités géographiques, tandis que l'efficacité de la normalisation automatique varie selon les outils utilisés. Enfin, Ludovic Moncla et son équipe, dans leur étude intitulée \textit{Automated Geoparsing of Paris Street Names in 19th Century Novels} \cite{10.1145/3149858.3149859}, ont proposé une solution automatisée pour l'analyse toponymique des romans historiques, en combinant géoparsing, NER et outils de textométrie.

\par
Contrairement aux textes littéraires conventionnels tels que les romans, la poésie, en tant que forme textuelle particulière, pose des défis spécifiques pour la NER en raison de sa liberté structurelle et de son ambiguïté expressive. Les travaux récents de Foley, J. (\textit{Poetry: Identification, entity recognition, and retrieval}\cite{foley2019poetry}) ont proposé des méthodes de reconnaissance d'entités adaptées au format poétique et ont constitué un vaste corpus de recherche dédié. Par ailleurs, Zhou et al., dans leur étude \textit{Named Entity Recognition of Ancient Poems Based on Albert‐BiLSTM‐MHA‐CRF Model}\cite{zhou2022named}, ont développé un modèle combinant Albert (modèle pré-entraîné) avec des mécanismes BiLSTM-MHA-CRF, permettant une extraction efficace de vecteurs de caractères pour améliorer l'identification des entités dans les poèmes classiques chinois. Leurs résultats expérimentaux ont atteint un score F1 de 97,17\,\%, démontrant l'efficacité de cette approche dans des contextes linguistiques complexes.

\par
Cependant, les recherches sur la NER appliquée aux textes poétiques restent limitées, malgré les avancées observées dans d'autres domaines littéraires multilingues. Les études existantes se concentrent principalement sur la recherche d'entités à granularité grossière et sur des langues autres que le français, comme le chinois. Pour la poésie française, et plus spécifiquement celle du XIX\up{e} siècle, aucune évaluation approfondie des performances de la NER n'a encore été menée, malgré ses particularités telles que les variations orthographiques historiques, l'ambiguïté des entités (toponymes personnifiés, noms propres symboliques) et le style linguistique distinctif. De plus, les recherches actuelles sur le \textit{Truecasing} et la normalisation automatique se limitent généralement à des textes standards comme les sorties OCR ou les transcriptions vocales. Il n'existe pas assez d’études systématiques qui explorent l'impact des stratégies de segmentation des phrases ou de rétablissement de la casse dans des textes poétiques, où ces éléments jouent pourtant un rôle crucial en raison de leur dépendance à la mise en forme, à la ponctuation et aux conventions typographiques.

\par
C'est dans cette perspective que la présente étude se concentrera sur un corpus de poésie française du XIX\up{e} siècle, afin d'évaluer l’adaptabilité des systèmes NER avancés actuels à ce type de textes. Pour mieux cerner les spécificités du genre poétique, nous mènerons une comparaison systématique avec des textes littéraires contemporains ainsi qu'avec d’autres formes versifiées, telles que les paroles de chansons. En parallèle, une exploration approfondie sera consacrée à l'analyse des stratégies de segmentation des phrases et des méthodes de rétablissement de la casse les mieux adaptées aux particularités linguistiques et stylistiques de la poésie, afin d'évaluer leur impact précis sur l'amélioration des performances en NER. 

À travers cette démarche, cette recherche poursuit un double objectif : d'une part, développer une approche technique plus fine pour la reconnaissance d'entités dans des styles littéraires spécifiques, notamment la poésie classique ; d'autre part, enrichir les perspectives des études NER appliquées aux textes français, en tenant compte de la diversité des genres littéraires.
