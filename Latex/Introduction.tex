La reconnaissance d'entités nommées (Named Entity Recognition, NER) est particulièrement cruciale pour de nombreuses applications de traitement automatique des langues, notamment dans le domaine de l'extraction d'informations. La plupart des recherches sur la reconnaissance d'entités nommées s'appuient sur des corpus textuels modernes utilisant un langage formel, généralement constitués de textes longs avec des structures phrastiques bien formées, une capitalisation standardisée et exempts de fautes d'orthographe. En revanche, les études portant spécifiquement sur les entités toponymiques dans les textes littéraires français restent relativement rares.

Des recherches antérieures ont démontré que l'extraction des noms de lieux dans les textes littéraires et leur représentation cartographique permettent de révéler le paysage littéraire caractéristique d'une langue. Comme le souligne l'étude \textit{Automated Geoparsing of Paris Street Names in 19th Century Novels\cite{10.1145/3149858.3149859}} :
\begin{quote}
    \itshape
    In literature, a spatial turn corresponds to a new interest in places and landscapes described in fictional texts and in maps as a good way to reveal the spatial structure of a narrative. Beyond a traditional use of literary mapping, cartography is now used to produce a novelistic representation of a place with the aim of supporting new interpretations of novels.
\end{quote}

Cependant, il est connu que si la ponctuation des romans français des XIX\up{e} et XX\up{e} siècles est généralement normative, certaines œuvres adoptant une forme poétique en sont dépourvues. Or, la ponctuation participe à la construction syntaxique des phrases et s'avère essentielle pour la compréhension contextuelle. Face à des textes poétiques ou autres dépourvus de ponctuation, certains modèles comme \texttt{spaCy} tendent à traiter la phrase entière comme une seule entité lexicale, extrayant parfois des fragments verbaux --- y compris des segments quasi-complets --- comme entités nominales, alors qu'il ne s'agit manifestement pas d'entités nommées.

L'étude \textit{POETRY: Identification, Entity Recognition, and Retrieval \cite{foley2019poetry}} précise à ce sujet :
\begin{quote}
    \itshape
    The traditional NLP task of named entity recognition (NER), where the goal is to label the spans in text that refer to real people, places, organizations and things. Traditional approaches to NER are unsuitable for unstructured documents like poetry because almost all state-of-the-art approaches depend on sentence boundaries and capitalization for efficiency and understanding. Naturally, poetry, like 'internet-speak' discourse, may not contain any punctuation or capitalization while still referring to real-world entities.
\end{quote}

Notre ambition est d'appliquer une approche similaire à ce genre littéraire particulier qu'est la poésie, afin d'étudier si la poésie française du XIX\up{e} siècle privilégie la représentation des campagnes françaises ou s'inscrit plutôt dans une veine orientaliste. Cette recherche évaluera les performances des systèmes NER les plus avancés sur un corpus de poésie française du XIX\up{e} siècle, en les comparant avec des corpus littéraires contemporains ou d'autres formes versifiées (comme les paroles de chansons). Nous examinerons également dans quelle mesure des méthodes améliorées de segmentation phrastique et de traitement de la capitalisation, plus adaptées aux particularités linguistiques, pourraient optimiser les performances des modèles de NER.

Nous organisons la suite de cette contribution de la façon suivante : dans la section~\ref{sec:review}, nous présentons une revue des travaux sur la reconnaissance d'entités nommées dans les textes anciens. La section~\ref{sec:corpus} décrit le corpus de poésie française que nous utilisons, ainsi que notre méthodologie et les outils employés. Les sections~\ref{sec:experiments} et~\ref{sec:results} exposent et analysent en détail les résultats obtenus. Enfin, la section~\ref{sec:conclusion} synthétise les conclusions de notre étude et propose des pistes pour de futures recherches.

